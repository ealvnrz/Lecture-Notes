\documentclass[10pt,spanish,ignorenonframetext,,aspectratio=149]{beamer}
\setbeamertemplate{caption}[numbered]
\setbeamertemplate{caption label separator}{: }
\setbeamercolor{caption name}{fg=normal text.fg}
\usepackage{lmodern}
\usepackage{amssymb,amsmath}
\usepackage{ifxetex,ifluatex}
\usepackage{fixltx2e} % provides \textsubscript
\ifnum 0\ifxetex 1\fi\ifluatex 1\fi=0 % if pdftex
  \usepackage[T1]{fontenc}
  \usepackage[utf8]{inputenc}
\else % if luatex or xelatex
  \ifxetex
    \usepackage{mathspec}
  \else
    \usepackage{fontspec}
  \fi
  \defaultfontfeatures{Ligatures=TeX,Scale=MatchLowercase}
  \newcommand{\euro}{€}
\fi
% use upquote if available, for straight quotes in verbatim environments
\IfFileExists{upquote.sty}{\usepackage{upquote}}{}
% use microtype if available
\IfFileExists{microtype.sty}{%
\usepackage{microtype}
\UseMicrotypeSet[protrusion]{basicmath} % disable protrusion for tt fonts
}{}
\ifxetex
  \usepackage{polyglossia}
  \setmainlanguage[]{}
\else
  \usepackage[shorthands=off,spanish]{babel}
\fi
\usepackage{color}
\usepackage{fancyvrb}
\newcommand{\VerbBar}{|}
\newcommand{\VERB}{\Verb[commandchars=\\\{\}]}
\DefineVerbatimEnvironment{Highlighting}{Verbatim}{commandchars=\\\{\}}
% Add ',fontsize=\small' for more characters per line
\usepackage{framed}
\definecolor{shadecolor}{RGB}{248,248,248}
\newenvironment{Shaded}{\begin{snugshade}}{\end{snugshade}}
\newcommand{\AlertTok}[1]{\textcolor[rgb]{0.94,0.16,0.16}{#1}}
\newcommand{\AnnotationTok}[1]{\textcolor[rgb]{0.56,0.35,0.01}{\textbf{\textit{#1}}}}
\newcommand{\AttributeTok}[1]{\textcolor[rgb]{0.77,0.63,0.00}{#1}}
\newcommand{\BaseNTok}[1]{\textcolor[rgb]{0.00,0.00,0.81}{#1}}
\newcommand{\BuiltInTok}[1]{#1}
\newcommand{\CharTok}[1]{\textcolor[rgb]{0.31,0.60,0.02}{#1}}
\newcommand{\CommentTok}[1]{\textcolor[rgb]{0.56,0.35,0.01}{\textit{#1}}}
\newcommand{\CommentVarTok}[1]{\textcolor[rgb]{0.56,0.35,0.01}{\textbf{\textit{#1}}}}
\newcommand{\ConstantTok}[1]{\textcolor[rgb]{0.00,0.00,0.00}{#1}}
\newcommand{\ControlFlowTok}[1]{\textcolor[rgb]{0.13,0.29,0.53}{\textbf{#1}}}
\newcommand{\DataTypeTok}[1]{\textcolor[rgb]{0.13,0.29,0.53}{#1}}
\newcommand{\DecValTok}[1]{\textcolor[rgb]{0.00,0.00,0.81}{#1}}
\newcommand{\DocumentationTok}[1]{\textcolor[rgb]{0.56,0.35,0.01}{\textbf{\textit{#1}}}}
\newcommand{\ErrorTok}[1]{\textcolor[rgb]{0.64,0.00,0.00}{\textbf{#1}}}
\newcommand{\ExtensionTok}[1]{#1}
\newcommand{\FloatTok}[1]{\textcolor[rgb]{0.00,0.00,0.81}{#1}}
\newcommand{\FunctionTok}[1]{\textcolor[rgb]{0.00,0.00,0.00}{#1}}
\newcommand{\ImportTok}[1]{#1}
\newcommand{\InformationTok}[1]{\textcolor[rgb]{0.56,0.35,0.01}{\textbf{\textit{#1}}}}
\newcommand{\KeywordTok}[1]{\textcolor[rgb]{0.13,0.29,0.53}{\textbf{#1}}}
\newcommand{\NormalTok}[1]{#1}
\newcommand{\OperatorTok}[1]{\textcolor[rgb]{0.81,0.36,0.00}{\textbf{#1}}}
\newcommand{\OtherTok}[1]{\textcolor[rgb]{0.56,0.35,0.01}{#1}}
\newcommand{\PreprocessorTok}[1]{\textcolor[rgb]{0.56,0.35,0.01}{\textit{#1}}}
\newcommand{\RegionMarkerTok}[1]{#1}
\newcommand{\SpecialCharTok}[1]{\textcolor[rgb]{0.00,0.00,0.00}{#1}}
\newcommand{\SpecialStringTok}[1]{\textcolor[rgb]{0.31,0.60,0.02}{#1}}
\newcommand{\StringTok}[1]{\textcolor[rgb]{0.31,0.60,0.02}{#1}}
\newcommand{\VariableTok}[1]{\textcolor[rgb]{0.00,0.00,0.00}{#1}}
\newcommand{\VerbatimStringTok}[1]{\textcolor[rgb]{0.31,0.60,0.02}{#1}}
\newcommand{\WarningTok}[1]{\textcolor[rgb]{0.56,0.35,0.01}{\textbf{\textit{#1}}}}
\usepackage{graphicx,grffile}
\makeatletter
\def\maxwidth{\ifdim\Gin@nat@width>\linewidth\linewidth\else\Gin@nat@width\fi}
\def\maxheight{\ifdim\Gin@nat@height>\textheight0.8\textheight\else\Gin@nat@height\fi}
\makeatother
% Scale images if necessary, so that they will not overflow the page
% margins by default, and it is still possible to overwrite the defaults
% using explicit options in \includegraphics[width, height, ...]{}
\setkeys{Gin}{width=\maxwidth,height=\maxheight,keepaspectratio}

% Comment these out if you don't want a slide with just the
% part/section/subsection/subsubsection title:
\AtBeginPart{
  \let\insertpartnumber\relax
  \let\partname\relax
  \frame{\partpage}
}
\AtBeginSection{
  \let\insertsectionnumber\relax
  \let\sectionname\relax
  \frame{\sectionpage}
}
\AtBeginSubsection{
  \let\insertsubsectionnumber\relax
  \let\subsectionname\relax
  \frame{\subsectionpage}
}

\setlength{\emergencystretch}{3em}  % prevent overfull lines
\providecommand{\tightlist}{%
  \setlength{\itemsep}{0pt}\setlength{\parskip}{0pt}}
\setcounter{secnumdepth}{0}

\title{Métodos multivariados}
\author{Eloy Alvarado Narváez}
\date{}

%% Here's everything I added.
%%--------------------------

\usepackage{graphicx}
\usepackage{rotating}
%\setbeamertemplate{caption}[numbered]
\usepackage{hyperref}
\usepackage{caption}
\usepackage[normalem]{ulem}
%\mode<presentation>
\usepackage{wasysym}
\usepackage{amsmath}
\usepackage{tikz}
\usepackage{svg}
\usepackage{tcolorbox}

% Get rid of navigation symbols.
%-------------------------------
\setbeamertemplate{navigation symbols}{}

% Optional institute tags and titlegraphic.
% Do feel free to change the titlegraphic if you don't want it as a Markdown field.
%----------------------------------------------------------------------------------
\institute{Instituto de Estadística \newline Universidad de Valparaíso}

% \titlegraphic{\includegraphics[width=0.3\paperwidth]{\string~/Dropbox/teaching/clemson-academic.png}} % <-- if you want to know what this looks like without it as a Markdown field. 
% -----------------------------------------------------------------------------------------------------
\titlegraphic{\includegraphics[width=0.3\paperwidth]{logo.png}}

% Some additional title page adjustments.
%----------------------------------------
\setbeamertemplate{title page}[]
%\date{}
\setbeamerfont{subtitle}{size=\small}

\setbeamercovered{transparent}

% Some optional colors. Change or add as you see fit.
%---------------------------------------------------
\definecolor{clemsonpurple}{HTML}{000000}
\definecolor{clemsonorange}{HTML}{F66733}
\definecolor{uiucblue}{HTML}{003C7D}
\definecolor{uiucorange}{HTML}{F47F24}

\definecolor{yellow}{HTML}{FFCC00}
\definecolor{blue}{HTML}{003399}
%\definecolor{black}{HTML}{000000}

% Some optional color adjustments to Beamer. Change as you see fit.
%------------------------------------------------------------------
\setbeamercolor{frametitle}{fg=black,bg=white}
\setbeamercolor{title}{fg=black,bg=white}
\setbeamercolor{local structure}{fg=black}
\setbeamercolor{section in toc}{fg=black,bg=white}
% \setbeamercolor{subsection in toc}{fg=clemsonorange,bg=white}
\setbeamercolor{footline}{fg=black!50, bg=white}
\setbeamercolor{block title}{fg=black,bg=white}


\let\Tiny=\tiny


% Sections and subsections should not get their own damn slide.
%--------------------------------------------------------------
\AtBeginPart{}
\AtBeginSection{}
\AtBeginSubsection{}
\AtBeginSubsubsection{}

% Suppress some of Markdown's weird default vertical spacing.
%------------------------------------------------------------
\setlength{\emergencystretch}{0em}  % prevent overfull lines
\setlength{\parskip}{0pt}


% Allow for those simple two-tone footlines I like. 
% Edit the colors as you see fit.
%--------------------------------------------------
\defbeamertemplate*{footline}{my footline}{%
    \ifnum\insertpagenumber=1
    \hbox{%
        \begin{beamercolorbox}[wd=\paperwidth,ht=.8ex,dp=1ex,center]{}%
      % empty environment to raise height
        \end{beamercolorbox}%
    }%
    \vskip0pt%
    \else%
        \Tiny{%
            \hfill%
		\vspace*{1pt}%
            \insertframenumber/\inserttotalframenumber \hspace*{0.1cm}%
            \newline%
            \color{blue}{\rule{\paperwidth}{0.4mm}}\newline%
            \color{yellow}{\rule{\paperwidth}{.4mm}}%
        }%
    \fi%
}

% Various cosmetic things, though I must confess I forget what exactly these do and why I included them.
%-------------------------------------------------------------------------------------------------------
\setbeamercolor{structure}{fg=blue}
\setbeamercolor{local structure}{parent=structure}
\setbeamercolor{item projected}{parent=item,use=item,fg=black,bg=white}
\setbeamercolor{enumerate item}{parent=item}

% Adjust some item elements. More cosmetic things.
%-------------------------------------------------
\setbeamertemplate{itemize item}{\color{black}$\bullet$}
\setbeamertemplate{itemize subitem}{\color{black}\scriptsize{$\bullet$}}
\setbeamertemplate{itemize/enumerate body end}{\vspace{.6\baselineskip}} % So I'm less inclined to use \medskip and \bigskip in Markdown.

% Automatically center images
% ---------------------------
% Note: this is for ![](image.png) images
% Use "fig.align = "center" for R chunks

\usepackage{etoolbox}

\AtBeginDocument{%
  \letcs\oig{@orig\string\includegraphics}%
  \renewcommand<>\includegraphics[2][]{%
    \only#3{%
      {\centering\oig[{#1}]{#2}\par}%
    }%
  }%
}

% I think I've moved to xelatex now. Here's some stuff for that.
% --------------------------------------------------------------
% I could customize/generalize this more but the truth is it works for my circumstances.

\ifxetex
\setbeamerfont{title}{family=\fontspec{Titillium Web}}
\setbeamerfont{frametitle}{family=\fontspec{Titillium Web}}
\usepackage[font=small,skip=0pt]{caption}
 \else
 \fi

% Okay, and begin the actual document...



\usepackage{tikz}
\usebackgroundtemplate{
  \tikz[overlay,remember picture] 
  \node[opacity=0.3, at=(current page.south west),anchor=south west,inner sep=10pt]{
    \includegraphics[width=1.5cm]{logo}};
}
\begin{document}
\frame{\titlepage}



\hypertarget{introducciuxf3n}{%
\section{Introducción}\label{introducciuxf3n}}

\begin{frame}{Introducción}
\begin{itemize}
\item
  La mayoría de estudios de investigación requieren obtener
  observaciones con muchas variables diferentes.
\item
  El análisis multivariado (y por ende sus técnicas) es la disciplina
  encargada de obtener información de este tipo datos. Particularmente
  usando técnicas estadísticas.
\end{itemize}
\end{frame}

\begin{frame}
En general, podemos identificar varios objetivos que conllevan el uso de
técnicas multivariadas.

\begin{itemize}
\tightlist
\item
  Reducción de datos o simplificación estructural
\item
  Ordenar o agrupar
\item
  Investigación de dependencia entre las variables
\item
  Predicción
\item
  Construcción de hipótesis y testeo
\end{itemize}
\end{frame}

\hypertarget{descripciuxf3n-de-datos-multivariados}{%
\section{Descripción de datos
multivariados}\label{descripciuxf3n-de-datos-multivariados}}

\hypertarget{matriz-de-datos}{%
\subsection{Matriz de datos}\label{matriz-de-datos}}

\begin{frame}{Matriz de datos}
La matriz general de datos \((n \times p)\) con \(n\) objetos y \(p\)
variables puede ser escrita como:

\[\mathbf{X}=\left[\begin{array}{cccccc}
x_{11} & x_{12} & \cdots & x_{1 k} & \cdots & x_{1 p} \\
x_{21} & x_{22} & \cdots & x_{2 k} & \cdots & x_{2 p} \\
\vdots & \vdots & & \vdots & & \vdots \\
x_{j 1} & x_{j 2} & \cdots & x_{j k} & \cdots & x_{j p} \\
\vdots & \vdots & & \vdots & & \vdots \\
x_{n 1} & x_{n 2} & \cdots & x_{n k} & \cdots & x_{n p}
\end{array}\right]\]

Esta matriz \(\mathbf{X}\) le llamamos la \textbf{matriz de datos}.
\end{frame}

\hypertarget{estaduxedstica-descriptiva}{%
\subsection{Estadística descriptiva}\label{estaduxedstica-descriptiva}}

\begin{frame}{Estadística descriptiva}
\begin{itemize}
\tightlist
\item
  ¿Es posible extender las nociones de estadística descriptiva
  unidimensional a esta matriz de datos?
\end{itemize}
\end{frame}

\begin{frame}{Vector de medias y matriz de covarianza}
\protect\hypertarget{vector-de-medias-y-matriz-de-covarianza}{}
Una extensión natural de la media aritmética y varianza son las
siguientes:

La \textbf{media muestral} de la \(i-\)ésima variable es:

\[\overline{x}_i={1\over n} \sum_{r=1}^{n} x_{ri}\]

y la \textbf{varianza muestral} de la \(i-\)ésima variable:

\[s_{ii}={1\over n}\sum_{r=1}^{n} (x_{ri}-\overline{x}_i)^2=s_{i}^{2}\quad \text{con}\quad i=1,\ldots,p\]
\end{frame}

\begin{frame}
La covarianza muestral entre la \(i-\)ésima y la \(j-\)ésima variable
estará dado:

\[s_{ij}={1\over n}\sum_{r=1}^{n} (x_{ri}-\overline{x}_i)(x_{rj}-\overline{x}_j)=s_{i}^{2}\quad \text{con}\quad i=1,\ldots,p\]

Así, el \textbf{vector de medias} estará dado por:

\[\overline{\mathbf{x}}=\begin{bmatrix}
           \overline{x}_{1} \\
           \overline{x}_{2} \\
           \vdots \\
           \overline{x}_{p}
         \end{bmatrix}\]

Este vector representa el centro de gravedad de los puntos
\(x_{r\cdot},r=1,\ldots,n\). La matrix \((p\times p)\)

\[\mathbf{S}=(s_{ij})\] en donde sus elementos están dados por
\(s_{ii}\) y \(s_{ij}\) es llamada la \textbf{matrix de covarianza
muestral}
\end{frame}

\begin{frame}
Las expresiones anterior también puede ser expresadas en notación
matricial:

\[\overline{\mathbf{x}}={1\over n}\sum_{r=1}^{n} \mathbf{x}_{r}={1\over n} \mathbf{X}'\mathbf{1}\]

En donde \(\mathbf{1}\) es un vector columna de \(n\) unos. Además,

\[s_{ij}={1\over n}\sum_{r=1}^{n} x_{ri}x_{rj}-\overline{x}_i \overline{x}_j\]

Por lo que

\[\mathbf{S}={1\over n}\sum_{r=1}^{n} (\mathbf{x}_r -\overline{\mathbf{x}})(\mathbf{x}_r -\overline{\mathbf{x}})'={1\over n} \sum_{r=1}^{n} \mathbf{x}_r \mathbf{x}_{r}' - \overline{\mathbf{x}}\overline{\mathbf{x}}'\]
\end{frame}

\begin{frame}
Equivalentemente,

\[\mathbf{S}={1\over n}\mathbf{X}'\mathbf{X}-\overline{\mathbf{x}}\overline{\mathbf{x}}'={1\over n}\left(\mathbf{X}'\mathbf{X}-{1\over n}\mathbf{X}'\mathbf{1}\mathbf{1}'\mathbf{X}\right)\]
\end{frame}

\begin{frame}{Matriz de centrado}
\protect\hypertarget{matriz-de-centrado}{}
Denotamos \(\mathbf{H}\) como la matriz de centrado, definida como:

\[\mathbf{H}=\mathbf{I}-{1\over n}\mathbf{1}\mathbf{1}'\]

Luego, podemos usar para reescribir

\[\mathbf{S}={1\over n}\mathbf{X}'\mathbf{H}\mathbf{X}\]

que es una forma conveniente de representar la matriz de covarianza.
\end{frame}

\begin{frame}
Debido a que la matriz \(\mathbf{H}\) es simétrica e idempotente
(\(\mathbf{H}=\mathbf{H}',\mathbf{H}=\mathbf{H}^2\)), sigue que para
cualquier \(p-\)vector \(a\), se tiene:

\[\mathbf{a}'\mathbf{S}\mathbf{a}={1\over n}\mathbf{a}'\mathbf{X}'\mathbf{H}'\mathbf{H}\mathbf{X}\mathbf{a}={1\over n}\mathbf{y}'\mathbf{y}\geq 0\]

donde \(\mathbf{y}=\mathbf{H}\mathbf{X}\mathbf{a}\). Por lo que la
matriz de covarianza \(\mathbf{S}\) es semi-definida positiva.

Al igual que en estadística univariada, usualmente es conveniente
definir la matriz de covarianza con divisor \(n-1\) en vez de \(n\).
Así,

\[\mathbf{S}_u={1\over n-1}\mathbf{X}'\mathbf{H}\mathbf{X}={n \over n-1} \mathbf{S}\]
\end{frame}

\begin{frame}
Si los datos forman una muestra aleatoria desde una distribución
multivariada, con segundo momento finito, entonces \(\mathbf{S}_u\) es
un estimador insesgado de la matriz de covarianza poblacional.

La matriz

\[\mathbf{M}=\sum_{r=1}^{n}\mathbf{x}_r\mathbf{x}_r'=\mathbf{X}'\mathbf{X}\]

es llamada la \textbf{matriz de suma de cuadrados y productos cruzados}.
\end{frame}

\hypertarget{matriz-de-correlaciuxf3n}{%
\subsection{Matriz de correlación}\label{matriz-de-correlaciuxf3n}}

\begin{frame}{Matriz de correlación}
El coeficiente de correlación muestral entre la \(i-\)ésima y
\(j-\)ésima variable, está dado:

\[r_{ij}={s_{ij}\over (s_i s_j)}\]

Así, la matriz definida como:

\[\mathbf{R}=(r_{ij})\]

con \(r_{ii}=1\) es llamada la \textbf{matriz de correlación muestral}.
Si \(\mathbf{R}=\mathbf{I}\), decimos que las variables no están
correlacionadas. Si definimos \(\mathbf{D}=\text{diag}(s_i)\), entonces

\[\mathbf{R}=\mathbf{D}^{-1}\mathbf{S}\mathbf{D}^{-1}, \quad \mathbf{S}=\mathbf{D}\mathbf{R}\mathbf{D}\]
\end{frame}

\begin{frame}{Ejercicio}
\protect\hypertarget{ejercicio}{}
Se desea investigar sobre las ventas realizadas en una librería, para
ello se tienen cuatro boletas, en ella se especifica (entre otras cosas)
el número de libros comprados y el precio total pagado, como sigue:

\begin{align*}
\text{Monto total (USD)}:\quad && 42 && 52 && 48 && 58 \\
\text{Cantidad de libros}:\quad && 4 && 5 && 4 && 3
\end{align*}

Usando la notación presentada, escriba las matrices:

\[\mathbf{X}\quad \overline{\mathbf{x}}\quad \mathbf{S}\quad \mathbf{R}\]
\end{frame}

\hypertarget{combinaciones-lineales}{%
\section{Combinaciones lineales}\label{combinaciones-lineales}}

\begin{frame}{Combinaciones lineales}
Tomar combinaciones lineales de las variables es una de las herramientas
más importantes del análisis multivariado. Un par de combinaciones
lineales escogidas de manera correcta pueden entregar más información
que una multiplicidad de variables originales, usualmente debido a que
la dimensionalidad se reduce.

Las combinaciones lineales también pueden simplificar la estructura de
la matriz de covarianza, por lo que interpretar los datos se hace más
sencillo.
\end{frame}

\begin{frame}
Consideremos la siguiente combinación lineal

\[y_r=a_1x_{r1}+\dots+a_p x_{np}\]

donde \(a_1,\dots,a_p\) son dados. Sabemos que la media

\[\overline{y}={1\over n} \mathbf{a}' \sum_{r=1}^{n} \mathbf{x}_r=\mathbf{a}'\overline{\mathbf{x}}\]

y la varianza está dada por

\[s_{y}^{2}={1\over n} \sum_{r=1}^{n} (y_r-\overline{y})^2={1\over n} \sum_{r=1}^{n} \mathbf{a}'(\mathbf{x}_r-\overline{\mathbf{x}})(\mathbf{x}_r-\overline{\mathbf{x}})'\mathbf{a}=\mathbf{a}'\mathbf{S}\mathbf{a}\]
\end{frame}

\begin{frame}
En general, estaremos interesados en una transformación lineal
\(q-\)dimensional

\[\mathbf{y}_r=\mathbf{Ax}+\mathbf{b}\quad r=1,\dots,n\]

que también puede ser escrito como

\[\mathbf{Y}=\mathbf{XA}'+\mathbf{1b}'\] donde \(\mathbf{A}\) es una
matriz \((q\times p)\) y \(\mathbf{b}\) es un \(q-\)vector. Usualmente
\(q\leq p\).
\end{frame}

\begin{frame}
El vector de medias y la matriz de covarianza del nuevo objeto
\(\mathbf{y}\), estarán dados por:

\[\overline{\mathbf{y}}=\mathbf{A}\overline{\mathbf{x}}+\mathbf{b}\] y,
\[\mathbf{S}_y={1\over n}\sum_{r=1}^{n}(\mathbf{y}_r-\overline{\mathbf{y}})(\mathbf{y}_r-\overline{\mathbf{y}})'=\mathbf{ASA}'\]

Si \(\mathbf{A}\) es una matriz no singular (en particular, \(q=p\)),
entonces

\[\mathbf{S}=\mathbf{A}^{-1}\mathbf{S}_y(\mathbf{A}')^{-1}\]

En lo que sigue daremos varios ejemplos de transformaciones lineales que
usaremos más adelante.
\end{frame}

\begin{frame}{Transformación de escala}
\protect\hypertarget{transformaciuxf3n-de-escala}{}
Sea \(y_r=\mathbf{D}^{-1}(\mathbf{x}_r-\overline{\mathbf{x}})\) , con
\(r=1,\dots,n\), donde \(\mathbf{D}=diag(s_i)\). Esta transformación
escala cada variable a varianza unitaria por lo que elimina la
arbitrariedad de la escala de medición escogida.

Por ejemplo, si \(\mathbf{x}_{(1)}\) mide longitud, entonces \(y_{(1)}\)
será la misma independiente si \(\mathbf{x}_{(1)}\) se mide en pulgadas
o metros.

Notar que \(\mathbf{S}_y=\mathbf{R}\).
\end{frame}

\hypertarget{transformaciuxf3n-de-mahalanobis}{%
\subsection{Transformación de
Mahalanobis}\label{transformaciuxf3n-de-mahalanobis}}

\begin{frame}{Transformación de Mahalanobis}
Sea \(\mathbf{S}>0\) entonces \(\mathbf{S}^{-1}\) tiene una única raíz
cuadrada simétrica definida positiva \(\mathbf{S}^{-1/2}\). La
transformación de Mahalanobis está definida por

\[\mathbf{z}_r=\mathbf{S}^{-1} (\mathbf{x}_{r} -\overline{\mathbf{x}}), \quad r=1,\dots,n\]

Así, \(\mathbf{S}_z=\mathbf{I}\), por lo que esta transformación elimina
la correlación entre las variables y estandariza la varianza de cada
variable.
\end{frame}

\hypertarget{transformaciuxf3n-de-componentes-principales}{%
\subsection{Transformación de componentes
principales}\label{transformaciuxf3n-de-componentes-principales}}

\begin{frame}{Transformación de componentes principales}
Por el teorema de descomposición espectral, la matriz de covarianza
\(\mathbf{S}\) puede ser escrita como

\[
\mathbf{S}=\mathbf{GLG}'
\]

donde \(\mathbf{G}\) es una matriz ortogonal y \(\mathbf{L}\) es una
matriz diagonal de los valores propies de \(\mathbf{S}\),
\(l_1\geq l_2\geq\dots\geq l_p\geq 0\). La transformación de componentes
principales está definida por la \emph{rotación}

\[\mathbf{w}_r=\mathbf{G}'(\mathbf{x}_r-\overline{\mathbf{x}}), \quad r=1,\dots,n\]
Debido a que \(\mathbf{S}_w=\mathbf{G}'\mathbf{S}\mathbf{G}=\mathbf{L}\)
es diagonal, las columnas de \(\mathbf{W}\), llamadas
\textbf{componentes principales} representan un combinación lineal no
correlacionada de las variables.

En la práctica, se desea resumir la mayoría de la variabilidad en los
datos usando sólo las componentes principales de mayor varianza,
reduciendo así la dimensionalidad.
\end{frame}

\hypertarget{gruxe1ficos-multivariantes}{%
\section{Gráficos multivariantes}\label{gruxe1ficos-multivariantes}}

\begin{frame}{Gráficos multivariantes}
En general el proceso de visualización de datos bivariados no es
complicado, pero para 3 o más variables graficar se dificulta
enormemente. Sin embargo, existen propuestas de gráficos para poder
conllevar esta problemática.
\end{frame}

\begin{frame}
\begin{figure}
\centering
\includegraphics{figs/mult_plot.png}
\caption{Representación de 4 dimensiones}
\end{figure}
\end{frame}

\begin{frame}
Describiremos 4 metodologías para visualizar \(p\) dimensiones

\begin{enumerate}
\tightlist
\item
  \textbf{Perfiles}: Representa cada punto por \(p\) barras verticales,
  en donde las alturas de las barras representan el valor de las
  variables.
\item
  \textbf{Estrellas}: Retrata el valor de cada variable normalizada como
  un punto a lo largo de una línea entre el centro y el perímetro de un
  círculo. Los puntos usualmente se unen para formar un polígono
\item
  \textbf{Glifos}: Son círculos de tamaño fijo con líneas cuyas
  longitudes representan el valor de las variables.
\item
  \textbf{Caras}: Representa cada variable variable como una
  característica en una cara, como la longitud de la nariz, tamaño de
  los ojos, forma de los ojos, etc.
\item
  \textbf{Cajas}: Representa cada variable como la longitud de una
  dimensión de una caja. Para más de tres variables, las dimensiones son
  particionadas en segmentos.
\end{enumerate}
\end{frame}

\begin{frame}
\begin{figure}
\centering
\includegraphics{figs/mult_plot2.png}
\caption{Graficos multivariantes}
\end{figure}
\end{frame}

\hypertarget{ejemplos-en-r}{%
\subsection{Ejemplos en R}\label{ejemplos-en-r}}

\begin{frame}[fragile]{Ejemplos en R}
\begin{Shaded}
\begin{Highlighting}[]
\FunctionTok{library}\NormalTok{(carData)}
\FunctionTok{library}\NormalTok{(ggplot2)}
\FunctionTok{data}\NormalTok{(Salaries, }\AttributeTok{package=}\StringTok{"carData"}\NormalTok{)}
\FunctionTok{head}\NormalTok{(Salaries)}
\end{Highlighting}
\end{Shaded}

\begin{verbatim}
##        rank discipline yrs.since.phd yrs.service  sex salary
## 1      Prof          B            19          18 Male 139750
## 2      Prof          B            20          16 Male 173200
## 3  AsstProf          B             4           3 Male  79750
## 4      Prof          B            45          39 Male 115000
## 5      Prof          B            40          41 Male 141500
## 6 AssocProf          B             6           6 Male  97000
\end{verbatim}
\end{frame}

\begin{frame}[fragile]
\begin{Shaded}
\begin{Highlighting}[]
\NormalTok{plot1}\OtherTok{\textless{}{-}}\FunctionTok{ggplot}\NormalTok{(Salaries, }
       \FunctionTok{aes}\NormalTok{(}\AttributeTok{x =}\NormalTok{ yrs.since.phd, }
           \AttributeTok{y =}\NormalTok{ salary)) }\SpecialCharTok{+}
  \FunctionTok{geom\_point}\NormalTok{() }\SpecialCharTok{+} 
  \FunctionTok{labs}\NormalTok{(}\AttributeTok{title =} \StringTok{"Salario de Académicos por años desde su doctorado"}\NormalTok{)}
\end{Highlighting}
\end{Shaded}
\end{frame}

\begin{frame}[fragile]
\begin{Shaded}
\begin{Highlighting}[]
\NormalTok{plot1}
\end{Highlighting}
\end{Shaded}

\includegraphics{figs/unnamed-chunk-3.pdf}
\end{frame}

\begin{frame}[fragile]
\begin{Shaded}
\begin{Highlighting}[]
\NormalTok{plot2}\OtherTok{\textless{}{-}}\FunctionTok{ggplot}\NormalTok{(Salaries, }\FunctionTok{aes}\NormalTok{(}\AttributeTok{x =}\NormalTok{ yrs.since.phd, }
                     \AttributeTok{y =}\NormalTok{ salary, }
                     \AttributeTok{color=}\NormalTok{rank)) }\SpecialCharTok{+}
  \FunctionTok{geom\_point}\NormalTok{() }\SpecialCharTok{+}
  \FunctionTok{labs}\NormalTok{(}\AttributeTok{title =} \StringTok{"Salaries de académicos por rango y}
\StringTok{       años desde su doctorado"}\NormalTok{)}
\end{Highlighting}
\end{Shaded}
\end{frame}

\begin{frame}[fragile]
\begin{Shaded}
\begin{Highlighting}[]
\NormalTok{plot2}
\end{Highlighting}
\end{Shaded}

\includegraphics{figs/unnamed-chunk-5.pdf}
\end{frame}

\begin{frame}[fragile]
\begin{Shaded}
\begin{Highlighting}[]
\NormalTok{plot3}\OtherTok{\textless{}{-}}\FunctionTok{ggplot}\NormalTok{(Salaries, }
       \FunctionTok{aes}\NormalTok{(}\AttributeTok{x =}\NormalTok{ yrs.since.phd, }
           \AttributeTok{y =}\NormalTok{ salary, }
           \AttributeTok{color =}\NormalTok{ rank, }
           \AttributeTok{shape =}\NormalTok{ sex)) }\SpecialCharTok{+}  \FunctionTok{geom\_point}\NormalTok{(}\AttributeTok{size =} \DecValTok{3}\NormalTok{,}\AttributeTok{alpha =}\NormalTok{ .}\DecValTok{6}\NormalTok{) }\SpecialCharTok{+}
  \FunctionTok{labs}\NormalTok{(}\AttributeTok{title =} \StringTok{"Salario de académicos por rango, sexo y}
\StringTok{       años desde su doctorado"}\NormalTok{)}
\end{Highlighting}
\end{Shaded}
\end{frame}

\begin{frame}[fragile]
\begin{Shaded}
\begin{Highlighting}[]
\NormalTok{plot3}
\end{Highlighting}
\end{Shaded}

\includegraphics{figs/unnamed-chunk-7.pdf}
\end{frame}

\begin{frame}[fragile]
\begin{Shaded}
\begin{Highlighting}[]
\NormalTok{plot4}\OtherTok{\textless{}{-}}\FunctionTok{ggplot}\NormalTok{(Salaries, }
       \FunctionTok{aes}\NormalTok{(}\AttributeTok{x =}\NormalTok{ yrs.since.phd, }
           \AttributeTok{y =}\NormalTok{ salary, }
           \AttributeTok{color =}\NormalTok{ rank, }
           \AttributeTok{size =}\NormalTok{ yrs.service)) }\SpecialCharTok{+}
  \FunctionTok{geom\_point}\NormalTok{(}\AttributeTok{alpha =}\NormalTok{ .}\DecValTok{6}\NormalTok{) }\SpecialCharTok{+}
  \FunctionTok{labs}\NormalTok{(}\AttributeTok{title =} \StringTok{"Salario de académicos por rango, años de servicio }
\StringTok{       y años desde su doctorado"}\NormalTok{)}
\end{Highlighting}
\end{Shaded}
\end{frame}

\begin{frame}[fragile]
\begin{Shaded}
\begin{Highlighting}[]
\NormalTok{plot4}
\end{Highlighting}
\end{Shaded}

\includegraphics{figs/unnamed-chunk-9.pdf}
\end{frame}

\begin{frame}[fragile]
\begin{Shaded}
\begin{Highlighting}[]
\FunctionTok{library}\NormalTok{(GGally)}
\FunctionTok{ggpairs}\NormalTok{(iris[,}\SpecialCharTok{{-}}\DecValTok{5}\NormalTok{])}\SpecialCharTok{+} \FunctionTok{theme\_bw}\NormalTok{()}
\end{Highlighting}
\end{Shaded}

\includegraphics{figs/unnamed-chunk-10.pdf}
\end{frame}

\hypertarget{medidas-multivariadas-de-curtosis-y-asimetruxeda}{%
\subsection{Medidas multivariadas de curtosis y
asimetría}\label{medidas-multivariadas-de-curtosis-y-asimetruxeda}}

\begin{frame}{Medidas multivariadas de curtosis y asimetría}
Si bien hemos visto medidas de estadística descriptiva, estas fueron
basadas en los dos primeros momentos. En el caso multivariado, queremos
tener expresiones para la curtosis y asimetría al igual que
\(b_1=m_{3}^{3} / s^6\) y \(b_2=m_4/s^4\) que son las medidas de
asimetría y curtosis para el caso univariado.

Si usamos las funciones invariantes

\[
g_{rs}=(\mathbf{x_r-\overline{x})'S^{-1}(x_s-\overline{x})}
\]

Se pueden definir medidas de curtosis y asimetría multivariada de la
forma:

\[
b_{1,p}=\dfrac{1}{n^2}\sum_{r,s=1}^{n} g_{rs}^{3}
\]

y,

\[
b_{2,p}=\dfrac{1}{n}\sum_{r=1}^{n} g_{rr}^{2}
\]
\end{frame}

\begin{frame}{Taller \#1}
\protect\hypertarget{taller-1}{}
Ejercicios (Libro Mardia)

\begin{itemize}
\item
  1.4.1
\item
  1.51
\item
  1.52
\item
  1.53
\item
  1.54
\end{itemize}

Ejercicios (Libro Rencher) Usando R, resolver:

\begin{itemize}
\item
  3.18
\item
  3.22
\end{itemize}
\end{frame}

\begin{frame}
\end{frame}


%\section[]{}
%\frame{\small \frametitle{Table of Contents}
%\tableofcontents}
\end{document}
