\documentclass{article}
\usepackage[margin=1.5cm]{geometry}
\usepackage{graphicx}
\usepackage{fancyhdr}
\usepackage{enumerate}
\usepackage{epsfig}
\usepackage{graphics,color, epstopdf}
\setlength{\headheight}{70pt} 
\setlength{\headsep}{20pt} 
\pagestyle{fancy}
\lhead{\includegraphics[width=3cm,height=1.5cm]{uv_logo_alta_rgba.png}}
\chead{Universidad de Valpara\'iso \\ Facultad de Ciencias \\ Instituto de Estad\'istica }
\rhead{Profesor: Eloy Alvarado}
\begin{document}
\begin{center}
\textbf{Ejercicios estadística descriptiva}
\end{center}
\begin{enumerate}
\item Las siguientes observaciones corresponden a la magnitud de sismos ocurridos en California, seg\'un su medici\'on en escala de Richter:  \\
\setlength{\tabcolsep}{17pt}
\begin{table}[h]
\begin{center}
\begin{tabular}{ccllllllll}
1.0 & 8.3 & 3.1 & 1.1 & 5.1 & 2.0 & 1.9 & 6.3 & 1.4 & 1.3 \\
3.3 & 2.2 & 2.3 & 2.1 & 2.1 & 1.4 & 2.0 & 2.4 & 3.0 & 4.1 \\
5.0 & 2.2 & 1.2 & 7.7 & 1.5  &     &     &     &     &    
\end{tabular}
\end{center}
\end{table}
\begin{enumerate}[a)]
\item Identifique y clasifique la variable estudiada.
\item Realice un diagrama de tallo y hoja.
\item Proponga y calcule una medida de tendencia central adecuada.
\end{enumerate}
\item Una constructora desea conocer el nivel de compromiso alcanzado por sus trabajadores. Para ello analiza dos variables de inter\'es, el nivel de ingresos que percibe cada trabajador y la cantidad de licencias m\'edicas que presentan. A trav\'es de un estudio realizado se obtuvieron los siguientes resultados: \\

\begin{center}
  \begin{tabular}{|c|c|c|c|c|c|c|}
  \hline
    Ingreso (M\$) & $m_{i}$ & $n_{i}$ & $N_{i}$ & $f_{i}$ & $F_{i}$\\ \hline
    [\hfill -\hfill[ &  & 3 &  &  & \\ \hline
    [\hfill -\hfill[ &  &  & 17 &  & \\ \hline
    [\hfill -\hfill[ &  & 30 &  &  & \\ \hline
    [\hfill -\hfill[ &  &  &  &  & 0.95\\ \hline
    [\hfill -\hfill[ &  &  &  &  & \\ 

    \hline
  \end{tabular}
  \\\
  \vspace{15pt}
   
\setlength{\tabcolsep}{7pt}
  \begin{tabular}{ccccccccccccccc}
  \hline
  \multicolumn{15}{c}{N\'umero de licencias} \\ \hline
  0&0&2&0&1&3&5&0&1&1&0&2&1&5&4 \\
  2&4&4&5&3&1&1&0&1&4&1&1&3&3&0 \\
  1&5&0&4&0&0&3&0&1&3&2&5&3&5&2 \\
  4&0&3&2&2&4&2&2&2&5&0&0&0&1&3 \\
  \hline
  \end{tabular}
\end{center}

\begin{enumerate}[a)]
\vspace{5pt}
\item Complete la tabla de ingresos, comenzando el intervalo 1 con 148.5 y terminando el intervalo 5 con 398.5.
\item Construya una tabla de frecuencias para la cantidad de licencias.
\item Calcule las medidas de tendencia central para cada caracter\'istica.
\item La empresa establece que existe un compromiso adecuado de los trabajadores si el ingreso que perciben se encuentra entre los \$236.272 y \$356.120. ?`Qu\'e porcentaje de trabajadores se encuentra en esta clasificaci\'on?

\newpage
\end{enumerate}
\item En una m\'aquina autom\'atica de enlatado y etiquetado de productos del mar, se producen detenciones por latas que ingresan en mala posici\'on y traban el sistema. Se registro el n\'umero de detenciones que ocurrieron durante el per\'iodo de 70 d\'ias consecutivos. Las mediciones son las siguientes: \\
\begin{center}
\setlength{\tabcolsep}{7pt}
  \begin{tabular}{cccccccccc}
  \hline
  \\
  0&0&2&0&0&0&3&3&0&0 \\
1&8&5&0&0&4&3&0&6&2\\
3&1&1&0&1&0&1&1&0&0\\
2&2&0&0&0&1&2&1&2&0\\
0&1&6&4&3&3&1&2&4&0\\
0&3&1&2&0&0&0&0&0&1\\
1&0&2&0&2&2&4&0&2&2\\
\\
  \hline
  \end{tabular}
\end{center}
\begin{enumerate}[a)]
\item Identifique la variable en estudio.
\item Calcule medidas de tendencia central y dispersi\'on.
\end{enumerate}
\item La siguiente informaci\'on corresponde a puntajes obtenidos en un test de inteligencia aplicado a dos grupos diferentes. Los resultados se muestran en sus respectivas tablas de frecuencia: \\
\begin{center}
\begin{tabular}{|c|c|}
\hline
\multicolumn{2}{|c|}{Grupo A} \\ \hline
Puntaje              & $n_i$      \\ \hline
{[}414 $-$ 473{[}      & 5      \\ \hline
{[}473 $-$ 532{[}      & 8      \\ \hline
{[}532 $-$ 591{[}      & 15     \\ \hline
{[}591 $-$ 650{[}      & 3      \\ \hline
{[}650 $-$ 709{[}      & 4      \\ \hline
{[}709 $-$ 768{[}      & 1      \\ \hline
\end{tabular}
\quad
\quad\quad\quad\quad\quad\quad\qquad
\begin{tabular}{|c|c|}
\hline
\multicolumn{2}{|c|}{Grupo B} \\ \hline
Puntaje              & $n_i$      \\ \hline
{[}586 $-$ 629{[}      & 7      \\ \hline
{[}629 $-$ 672{[}      & 11     \\ \hline
{[}672 $-$ 715{[}      & 10     \\ \hline
{[}715 $-$ 758{[}      & 7      \\ \hline
{[}758 $-$ 801{[}      & 2      \\ \hline
{[}801 $-$ 844{[}      & 1       \\ \hline
\end{tabular}
\end{center}

\begin{enumerate}[a)]
\item Defina las variables y clasif\'iquelas.
\item Calcule medidas de tendencia central, varianza y desviaci\'on est\'andar para ambos grupos.
\item Si el 20\% de los pacientes con puntajes m\'as bajos son considerados retardados, el 10\% de los pacientes con mayor puntaje son considerados superdotados y los dem\'as pacientes son considerados normales, determine los l\'imites en los puntajes de una persona normal para el grupo A.
\item ?`En cu\'al de los dos grupos la distribuci\'on de los puntajes son m\'as homog\'eneos?
\end{enumerate}
\end{enumerate}
\end{document}