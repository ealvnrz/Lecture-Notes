\documentclass{article}
\usepackage[margin=2cm]{geometry}
\usepackage{graphicx}
\usepackage{amsmath,mathtools}
\usepackage{fancyhdr}
\usepackage{enumerate}
\usepackage{amsfonts}
\usepackage[utf8]{inputenc}
\usepackage[spanish]{babel}
\usepackage{epsfig}
\usepackage{graphics,color, epstopdf}
\setlength{\headheight}{70pt} 
\setlength{\headsep}{20pt} 
\pagestyle{fancy}
\lhead{}
\chead{Universidad de Valparaíso \\ Facultad de Ciencias \\ Instituto de Estadística }
\rhead{Profesor: Eloy Alvarado}

\begin{document}
\begin{center}
\textbf{Ejercicios Adicionales, Variables Aleatorias Bivariadas}
\end{center}
\begin{enumerate}
\item La función de probabilidad conjunta de dos variables aleatorias discretas $X,Y$ está dada por:
$$f(x,y)=\begin{cases}
cxy \hspace{20pt} x=1,2,3 ; y=1,2,3\\
0 \hspace{20pt} \text{ e.o.c. }
\end{cases}$$
\begin{enumerate}
\item Hallar la constante $c$.
\item $\mathbb{P}(X=2, Y=3)$
\item $\mathbb{P}(1\leq X \leq 2 ,Y\leq 2)$
\item $\mathbb{P}(X\geq 2)$
\item $\mathbb{P}(Y<2)$
\item $\mathbb{P}(X=1)$
\item $\mathbb{P}(Y=3)$
\end{enumerate}
\item Sean $X,Y$ variables aleatorias continuas que tienen función de densidad conjunta:
$$f(x,y)=\begin{cases}
c(x^2+y^2)\hspace{20pt} 0\leq x \leq 1,0\leq y\leq 1\\
0 \hspace{20pt} \text{ e.o.c. }
\end{cases}$$
Determinar
\begin{enumerate}
\item la constante $c$.
\item $\mathbb{P}\left(X < \dfrac{1}{2}, Y>\dfrac{1}{2}\right)$
\item $\mathbb{P}\left(\dfrac{1}{4}<X<\dfrac{3}{4}\right)$
\item $\mathbb{P}\left(Y < \dfrac{1}{2}\right)$
\item Si $X,Y$ son independientes.
\end{enumerate}
\item Si:
$$f(x,y)=\begin{cases}
x+y \hspace{20pt} 0\leq x \leq 1, 0 \leq y \leq 1\\
0 \hspace{20pt} \text{ e.o.c. }
\end{cases}$$
Hallar la función de densidad condicional de:
\begin{enumerate}
\item $X$ dada $Y$
\item $Y$ dada $X$
\end{enumerate}
\item Sea 
$$f(x,y)=\begin{cases}
\exp -(x+y) \hspace{20pt} x\geq 0, y\geq 0\\
0 \hspace{20pt} \text{ e.o.c. }
\end{cases}$$
La función de densidad conjunta de $X,Y$. Hallar la función de densidad condicional de:
\begin{enumerate}
\item $X$ dada $Y$
\item $Y$ dada $X$
\end{enumerate}

\end{enumerate}
\end{document}