\documentclass{article}
\usepackage[margin=2cm]{geometry}
\usepackage{graphicx}
\usepackage{fancyhdr}
\usepackage{enumerate}
\usepackage{epsfig}
\usepackage[utf8]{inputenc}
\usepackage[spanish]{babel}
\usepackage{graphics,color, epstopdf,amssymb}
\setlength{\headheight}{70pt} 
\setlength{\headsep}{20pt} 
\pagestyle{fancy}
\lhead{}
\chead{Universidad de Valpara\'iso \\ Facultad de Ciencias \\ Instituto de Estad\'istica }
\rhead{Profesor: Eloy Alvarado}

\begin{document}
\begin{center}
\textbf{Ejercicios Distribuciones}
\end{center}
\begin{enumerate}
\item En muchas de las Universidades Estadounidenses los postulantes a programas de postgrado requieren presentar sus resultados de una prueba estandarizada de matemáticas llamada GMAT(\textit{Graduate Management Admission Test}). Los puntajes de esta prueba son aproximadamente normales con media $527$ y desviación estándar $112$. 
\begin{enumerate}
\item ¿Cuál es la probabilidad que un estudiante haya obtenido un puntaje sobre 500 en el exámen?
\item ¿Qué puntaje debe obtener un estudiante para estar en el $5\%$ superior?
\end{enumerate}
\item La duración del embarazo humano desde su concepción hasta el nacimiento es aproximadamente normal con media $266$ y desviación estandar $16$ días.
\begin{enumerate}
\item ¿Qué proporción de todos los embarazos durará entre $240$ y $270$ días?
\item ¿Cuantos días debe durar un embarazo para que esté en el percentil 70?
\end{enumerate}
\item En promedio, el número de acres quemados debido a incendios forestales en \textit{New Mexico} es de $4300$ acres por año con desviación estándar $750$ acres. Si asumimos que el número de acres quedamos es normal.
\begin{enumerate}
\item ¿Cuál es la probabilidad que entre $2500$ y $4200$ acres serán quemados en cualquier año?
\item ¿Qué número de acres quemados corresponde al percentil 38?
\end{enumerate}
\item Una cadena de teatros realiza un estudio en sus usuarios para saber cuanto de su dinero es gastado en concesiones. El estudio revela que los gastos son aproximadamente normales con media $4.11$ dólares y desviación estándar $1.37$ dólares.
\begin{enumerate}
\item ¿Qué porcentaje de sus usuarios gastará menos de $3.00$ dólares en concesiones?
\item ¿Qué gasto correspende al $87\%$ superior?
\end{enumerate}
\item Suponga que el diámetro de cierto componente de un auto sigue una distribución normal de media $10$ mm y varianza $9$ mm$^2$. Encuentre la proporción de estos componentes que tiene diámetro mayor a $13.4$ mm.
\item Suponga que el peso de una caja de naranjas envasadas en cierta empresa distribuidora sigue una distribución normal de media $8$ kg y desviación estándar $1.5$.
\begin{enumerate}
\item ¿Cuál es la proporción de cajas de naranjas que pesa más de $11.5$ kg?
\item ¿Qué proporción de naranjas pesa menos de $8.7$ kg?
\item ¿Qué proporción de naranjas pesa menos de $5$ kg?
\item ¿Qué proporción de naranjas pesa entre $6.2$ y $7$ kg?
\item ¿Qué proporción de naranjas pesa entre $10.3$ y $14$ kg?
\item ¿Qué proporción de naranjas pesa entre $6.8$ y $8.9$ kg?
\item ¿Qué peso de caja de naranjas corresponde al percentil $80$?
\item ¿Qué peso de caja de naranjas corresponde al percentil $5$?
\item Encuentre el rango intercuartil de la variable peso de las cajas de naranja.
\end{enumerate}
\item MENSA es una organización cuyos miembros poseen IQ's \textit{(Intelligence quotient} en el $2\%$ superior de la población mundial.
\begin{enumerate}
\item Si los IQ's están normalmente distribuidos, con media 100 y desviación estándar 16.¿Cuál es el mínimo IQ requerido para ser admitido en la organización?
\item Si tres individuos son elegidos aleatoriamente de la población mundial, ¿Cuál es la probabilidad que las tres personas escogidas satisfagan los requirimientos para entrar a MENSA?
\end{enumerate}
\item Para evitar acusaciones de sexismo dentro de una universidad con la misma cantidad de alumnos hombres y mujeres, el profesor lanza una moneda para decidir si preguntar a un hombre o mujeres alguna pregunta relacionada con la clase. El profesor le preguntará a una mujer si sale sello. Suponga que el profesor lanza la moneda 1000 a lo largo de una semestre. Realice el cálculo de probabilidades utilizando la distribución normal y binomial.
\begin{enumerate}
\item ¿Cuál es la probabilidad que le pregunte a una mujer a lo menos 530 veces?
\item ¿Cuál es la probabilidad que le pregunte a una mujer a lo más 480 veces?
\item ¿Cuál es la probabilidad que le pregunte a una mujer exactamente 510 veces?
\end{enumerate}
\item Un proceso de manufactura produce semiconductores con una probabilidad de fallo de $6.3\%$. Suponga que los fallos son independientes entre sí. Si en una semana se producen $2000$ semiconductores.
\begin{enumerate}
\item Encuentre el número esperado de semiconductores fallidos.
\item Encuentre la desviación estándar del número de semiconductores fallidos-
\item Encuentre la probabilidad que se produzcan menos de $135$ semiconductores fallidos.
\end{enumerate}
\item $10\%$ de las partes de un computador producido por cierta empresa son fallidos.¿Cuál es la probabilidad que de una muestra de 10 partes, más de 3 sean fallidos?
\item En promedio, dos tornados de escala mayor ocurren en ciudades de los Estados Unidos cada año. ¿Cuál es la probabilidad que más de 5 tornados ocurran en ciudades el siguiente año?
\item Un laboratorio de redes de 20 computadores fueron atacados por un virus. El virus entra a cada computador con probabilidad $0.4$, independientemente de los otros computadores.
\begin{enumerate}
\item Encuentre la probabilidad que el virus entre en a lo menos 10 computadores.
\item Un administrador, revisa el laboratorio de computadores uno tras otro, para ver si las máquinas han sido infectadas. ¿Cuál es la probabilidad que deba revisar al menos 6 computadores para encontrar el primero infectado?
\end{enumerate}
\item En promedio, 1 computador de cada $800$ falla durante una tormenta de escala mayor. Cierta compañía tiene $4000$ computadores funcionando cuando el área de trabajo se ve afectada por la tormenta. \textit{Utilice aproximaciones adecuadas. }
\begin{enumerate}
\item Encuentre el valor esperado y varianza del número de computadores que fallan.
\item Encuentre la probabilidad que menos de $10$ computadores fallen.
\item Encuentre la probabilidad que exactamente $10$ computadores fallen.

\end{enumerate}
\end{enumerate}

\newpage
\begin{enumerate}
\item \begin{enumerate}
\item $0.5948$
\item $711.24$
\end{enumerate}
\item 
\begin{enumerate}
\item $0.5471$
\item $274.32$
\end{enumerate}
\item \begin{enumerate}
\item $0.4401$
\item $4067.5$
\end{enumerate}
\item \begin{enumerate}
\item $0.2090$
\item $2.56$
\end{enumerate}
\item $0.1292$
\item \begin{enumerate}
\item $0.0099$
\item $0.6808$
\item $0.0228$
\item $0.9808$
\item $0.1363$
\item $0.0630$
\item $0.5138$
\item $9.27$
\item $5.53$
\end{enumerate}
\item \begin{enumerate}
\item $132.88$
\item $0.000008$
\end{enumerate}
\item \begin{enumerate}
\item $0.0307$
\item $0.1093$
\item $0.0197$
\end{enumerate}
\item \begin{enumerate}
\item $126$
\item $10.87$
\item $0.7823$
\end{enumerate}
\item $0.0128$
\item $0.017$
\item \begin{enumerate}
\item $0.2447$
\item $0.0778$
\end{enumerate}
\item \begin{enumerate}
\item $\mathbb{E}(X)=5,\mathbb{V}(X)=4.994$
\item $0.968$
\item $0.018$
\end{enumerate}
\end{enumerate}
\end{document}