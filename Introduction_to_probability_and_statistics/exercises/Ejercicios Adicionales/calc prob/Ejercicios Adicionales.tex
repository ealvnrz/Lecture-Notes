\documentclass[a4paper]{article}
\usepackage[margin=2.5cm]{geometry}
\usepackage{graphicx}
\usepackage{fancyhdr}
\usepackage{enumerate}
\usepackage{epsfig}
\usepackage[utf8]{inputenc}
\usepackage[spanish]{babel}
\usepackage{graphics,color, epstopdf,amssymb,amsmath}
\setlength{\headheight}{70pt} 
\setlength{\headsep}{20pt} 
\pagestyle{fancy}
\lhead{}
\chead{Universidad de Valparaíso \\ Facultad de Ciencias \\ Instituto de Estadística }
\rhead{Profesor: Eloy Alvarado}

\begin{document}
\begin{center}
\textbf{Ejercicios}
\end{center}
\begin{enumerate}
\item Una carta se extrae aleatoriamente de una baraja de 52 cartas. Encontrar la probabilidad de que sea:
\begin{enumerate}[a)]
\item Un As.
\item Una jota de corazones.
\item Un tres de tréboles o un seis de diamantes.
\item Un Corazón
\item Cualquier palo excepto corazones.
\item Un diez o una pica.
\item Ni un cuatro ni un trébol.
\end{enumerate}
\item Una bola se extrae aleatoriamente de una caja que contiene 6 bolas rojas, 4 bolas blancas y 5 bolas azules. Determine la probabilidad de que sea:
\begin{enumerate}[a)]
\item Roja
\item Blanca
\item Azul
\item No roja
\item Roja o blanca
\end{enumerate}
\item Un dado honesto se lanza dos veces. Hallar la probabilidad de obtener 4,5 o 6 en el primer lanzamiento y 1,2,3 o 4 en el segundo lanzamiento.
\item Encontrar la probabilidad de no obtener un total de 7 u 11 en ninguno de los dos lanzamientos de un par de dados honrados.
\item Se extraen dos cartas de una baraja de 52 cartas. Hallar la probabilidad de que ambas sean ases si la carta:
\begin{enumerate}[a)]
\item Se reemplaza
\item No se reemplaza.
\end{enumerate}
\item Una caja contiene 4 bolas blancas y 2 bolas negras, otra caja contiene 3 bolas blancas y 5 bolas negras. Si se extrae una bola de cada caja, hallar la probabilidad de que:
\begin{enumerate}[a)]
\item Ambas sean blancas.
\item Ambas sean negras.
\item Una sea blanca y una negra.
\end{enumerate}
\item Una caja contiene 8 bolas rojas, 3 blancas y 9 azules. Si se extraen 3 bolas aleatoriamente sin reemplazamiento, determinar la probabilidad que:
\begin{enumerate}[a)]
\item Las 3 bolas sean rojas.
\item Las 3 bolas sean blancas.
\item 2 bolas sean rojas y 1 blanca.
\item Al menos 1 sea blanca.
\item Se extraiga una de cada color.
\item Las bolas sean extraídas en el orden: rojo, blanco y azul.
\end{enumerate}

\item Se extraen 5 cartas de una baraja de 52 cartas. Hallar la probabilidad de extraer:
\begin{enumerate}[a)]
\item 4 ases
\item 4 ases y un rey.
\item 3 diez y 2 jotas.
\item Un 9, 10, jota, reina y rey. En cualquier orden.
\item 3 de un palo y 2 de otro.
\item Al menos 1 as.

\end{enumerate}

\item Una variable aleatoria $X$ puede tomar los valores $30,40,50$ y $60$ con probabilidades $0.4,0.2,0.1$ y $0.3$
\begin{enumerate}[a)]
\item Represente en una tabla la función de masa de probabilidad y la función de distribución de probabilidad.
\item Determine las siguientes probabilidad
\begin{itemize}
\item $\mathbb{P}(X \leq 25)$
\item $\mathbb{P}(X \geq 60)$
\item $\mathbb{P}(X < 40)$
\item $\mathbb{P}(X > 40)$
\item $\mathbb{P}(30 \leq X \leq 60)$
\item $\mathbb{P}(30 \leq X < 60)$
\item $\mathbb{P}(30 < X \leq 60)$
\item $\mathbb{P}(30 < X < 60)$
\end{itemize}
\item Calcule la esperanza y varianza de $X$.

\end{enumerate}

\item Sea $X$ la variable aleatoria que representa el número de hijos por familia de una ciudad que tiene la siguiente función de probabilidad:
\begin{align*}
\mathbb{P}(X=x)=\begin{cases}
0.47\text{ si } x=0\\
0.3\text{ si } x=1\\
0.1\text{ si } x=2\\
0.06\text{ si } x=3\\
0.04\text{ si } x=4\\
0.02\text{ si } x=5\\
0.01\text{ si } x=6
\end{cases}
\end{align*}


\begin{enumerate}[a)]
\item Calcule la esperanza y varianza de la variable aleatoria $X$.
\item Si el gobierno paga $2000$ U.M por hijo y se define una nueva variable aleatoria como $Y=2000 X$. ¿Cuál es la distribución de probabilidad de Y?
\item Calcule la esperanza y varianza de la variable aleatoria $Y$.

\end{enumerate}

\item Complete la siguiente función de probabilidad, sabiendo que $\mathbb{E}(X)=1.8$
\begin{align*}
\mathbb{P}(X=x)=\begin{cases}
0.2\text{ si } x=0\\
a\text{ si } x=1\\
b\text{ si } x=2\\
0.3\text{ si } x=3\\
\end{cases}
\end{align*}







\end{enumerate}



\newpage
\begin{center}
\textbf{Respuestas}
\end{center}
\begin{enumerate}
\item 
\begin{enumerate}[a)]
\item $\dfrac{1}{13}$
\item $\dfrac{1}{52}$
\item $\dfrac{1}{26}$
\item $\dfrac{1}{4}$
\item $\dfrac{3}{4}$
\item $\dfrac{4}{13}$
\item $\dfrac{9}{13}$
\end{enumerate}
\item 
\begin{enumerate}[a)]
\item $\dfrac{2}{5}$
\item $\dfrac{4}{15}$
\item $\dfrac{1}{3}$
\item $\dfrac{3}{5}$
\item $\dfrac{2}{3}$

\end{enumerate}
\item $\dfrac{1}{3}$
\item $\dfrac{49}{81}$
\item 
\begin{enumerate}[a)]
\item $\dfrac{1}{169}$
\item $\dfrac{1}{221}$
\end{enumerate}
\item 
\begin{enumerate}[a)]
\item $\dfrac{1}{4}$
\item $\dfrac{5}{24}$
\item $\dfrac{13}{24}$
\end{enumerate}
\item 
\begin{enumerate}[a)]
\item $\dfrac{14}{285}$
\item $\dfrac{1}{1140}$
\item $\dfrac{7}{95}$
\item $\dfrac{23}{57}$
\item $\dfrac{18}{95}$
\item $\dfrac{3}{95}$
\end{enumerate}
\item 
\begin{enumerate}[a)]
\item $\dfrac{1}{54145}$
\item $\dfrac{1}{649740}$
\item $\dfrac{1}{108290}$
\item $\dfrac{64}{162435}$
\item $\dfrac{429}{4165}$
\item $\dfrac{18472}{54145}$

\end{enumerate}
\item 
\begin{enumerate}[a)]
\item[b)]
\begin{itemize}
\item $0$
\item $0.3$
\item $0.4$
\item $0.4$
\item $1$
\item $0.7$
\item $0.6$
\item $0.3$
\end{itemize}
\item[c)] $\mathbb{E}(X)=43,\mathbb{V}(X)=161$
\end{enumerate}
\item 
\begin{enumerate}[a)]
\item $\mathbb{E}(X)=1,\mathbb{V}(X)=1,74$
\item[c)] $\mathbb{E}(Y)=2000,\mathbb{V}(Y)=6,960,000$
\end{enumerate}
\item $a=0.1,b=0.4$

\end{enumerate}

\end{document}