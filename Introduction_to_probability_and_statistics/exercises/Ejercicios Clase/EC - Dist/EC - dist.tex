\documentclass{article}
\usepackage[margin=2cm]{geometry}
\usepackage{graphicx}
\usepackage{fancyhdr}
\usepackage{enumerate}
\usepackage{epsfig}
\usepackage[english]{babel}
\usepackage[utf8x]{inputenc}
\usepackage{amsmath}
\usepackage{graphics,color, epstopdf}
\setlength{\headheight}{70pt} 
\setlength{\headsep}{20pt} 
\pagestyle{fancy}
\lhead{}
\chead{Universidad de Valparaíso \\ Facultad de Ciencias \\ Instituto de Estadística }
\rhead{Profesor: Eloy Alvarado}

\begin{document}
\begin{center}
\textbf{Ejercicios Distribuciones Conocidas}
\end{center}
\begin{enumerate}
\item Durante los últimos años, se ha logrado establecer que el 30\% de los alumnos que ingresan por primera vez a cierta Universidad, reprueban todas las asignaturas de primer semestre.
Si, en el segundo semestre, se elige al azar a 15 alumnos que ingresaron el semestre anterior a la Universidad.
\begin{enumerate}[a)]
\item ¿Cuál es la probabilidad que sólo 5 de ellos hayan reprobado todas las asignaturas de primer semestre?



\item ¿Cuál es la probabilidad que a lo más 13 hayan reprobado todas las asignaturas de primer semestre?


\item ¿Cuál es la probabilidad de que 8 o más hayan reprobado todas las asignaturas?


Revisar pdf p.102

\end{enumerate}
\item Sólo el 30\% de la población de una gran ciudad, piensa que el sistema de transporte masivo es adecuado.

\begin{enumerate}[a)]
\item Si 20 personas son seleccionadas aleatoriamente de dicha población, encuentre la probabilidad de que 5 o menos, piensen que el sistema es adecuado.

Notar que el problema puede ser resuelto por medio de una distribución Binomial.

\[ P[X=k] = {n\choose k} p^k (1-p)^{n-k} \]
\[ P[X \leq 5] = \sum_{k=0}^5 {20\choose k} 0.3^k (0.7)^{20-k} \approx 0.41\]



\item Encuentre la probabilidad de que exactamente 6 piensen que el sistema es adecuado.

\[ P[X = 6] = {20\choose 6} 0.3^6 (0.7)^{20-6} \approx 0.19\]

\end{enumerate}
\item En un estudio invernal de una tienda, se determinó que un articulo se pide en promedio cinco veces por semana (de 5 días), de acuerdo a una distribución Poisson. ¿Cuál es la probabilidad de que en un día especifico, el articulo.


\begin{enumerate}[a)]
\item Se pida más de cinco veces.
\item No se pida.

Revisar pdf p.106

\end{enumerate}
\item Supongamos que los clientes llegan aleatoriamente a ser atendidos en un sistema de servicio, con una media de 2.6 clientes por hora, de acuerdo a una distribución Poisson.
El sistema tiene capacidad para atender a 5 personas por hora. Si más de cinco personas llegan durante una hora determinada no pueden recibir servicio.


\begin{enumerate}[a)]
\item En una hora dada, ¿cuál es la probabilidad que un cliente no pueda recibir servicio?

$$P(X < 5) = 1 - P(X\leq 5) \approx 5\% $$


\item ¿Cuál es la probabilidad de que durante un período determinado de una hora, el servidor esté ocioso?

$$P(X =0) \approx 7.4\%$$


\end{enumerate}

\item La duración de un laser semiconductor a potencia constante tiene una
distribución normal con media 7.000 horas y desviación estándar de 600 horas.


\begin{enumerate}[a)]
\item ¿Cuál es la probabilidad de que el laser falle antes de 5.000 horas?


\item ¿Cuál es la duración en horas excedida por el 99\% de los laser?

Revisar pdf p.116

\item Si se hace uso de tres laser en un producto y se supone que fallan de manera independiente. ¿Cuál es la probabilidad de que los tres sigan funcionando después de 6700 horas?


\end{enumerate}
\item La cantidad semanal que una compa\~n\'ia gasta en mantenimiento y reparaciones tiene una distribuci\'on normal con media US\$400 y desviaci\'on est\'andar US\$20.
El presupuesto para cubrir los gastos de reparaci\'on para la semana siguiente es US\$430.

\begin{enumerate}[a)]
\item ?`Cu\'al es la probabilidad que los costos reales sean mayores que la cantidad prevista?

\[ P[X \geq 430] = 1- P[X \leq 430] = 1- P[\dfrac{X-400}{20} \leq \dfrac{430-400}{20}]\]

\[ 1- P[X \leq 430] = 1- P[Z \leq 1.5] = 1- 0.9332 \approx 0.07\]

\item ?`Cu\'anto deber\'ia ser el presupuesto semanal para que sea sobrepasado s\'olo 1 vez de cada 1000?

\[ 0.001 = 1- 0.999 = 1 - P[Z\leq 3.09 ]\]

Despejamos $y$ de la siguiente ecuación 

\[ 3.09 = (y -400)/20 \]

Obteniendo el valor de $461.8$

\end{enumerate}
\item Se sabe que el di\'ametro de claveles de una variedad A producida por un floricultor, puede ser representada por una variable aleatoria cuya distribuci\'on es la normal con media 6.0 cm. y desviaci\'on est\'andar 0.3 cm


\begin{enumerate}[a)]
\item ?`Qu\'e porcentaje de claveles A del floricultor tiene un di\'ametro entre 6.5 y 6.8 cm.?

\[ P[ 6.5 \leq X \leq 6.8] =P[X \leq 6.8] - P[X \leq 6.5]  \approx 0.044 \]

\item Por no alcanzar el m\'inimo di\'ametro exigible por un pa\'is europeo, el floricultor s\'olo puede exportar el 25\% de di\'ametro superior de su producci\'on. ?`Cu\'al es ese di\'ametro m\'inimo exigido?

$$ 0.25 = 1- P[X \leq m ] $$
$$ P[X \leq m ] = 0.75 $$
$$ m= 6.20 $$

\item Suponga que si el floricultor desea entregar el 40\% de su producci\'on bajo las condiciones fijadas en b), debe hacer cambios para lograr aumentar su di\'ametro.
?`Cu\'anto deber\'ia ser el nuevo di\'ametro medio de los claveles? (Suponga que la desviaci\'on est\'andar se mantiene igual)


$$ 0.40 = 1- P[X \leq 6.20 ] $$
$$ P[X \leq 6.20 ] = 0.60 $$
$$ P[Z \leq \dfrac{6.2 - \overline{x}}{0.3}] = 0.6 $$
$$ 0.25 \leq \frac{6.2 -  \overline{x}}{0.3} $$
$$ \overline{x} \approx 6.12 $$
\end{enumerate}
\newpage

\item Un ingeniero ha determinado que el peso, en gramos, del residuo diario de cierto proceso industrial, es una variable aleatoria que sigue una distribuci\'on normal con media 180 gramos y desviaci\'on est\'andar 25 gramos.
En el per\'iodo de dos a\~nos, el 15\% de los d\'ias en que el residuo es menor, resulta que est\'a por debajo de un valor $P_{min}$, mientras que el 25\% de los d\'ias en que es mayor,
sobrepasa a un valor $P_{max}$. Obtenga:
\begin{enumerate}[a)]
\item $P_{min}$

$$ P[X \leq P_{min} ] = 0.15 $$
$$ P_{min} \approx 154.1 $$


\item $P_{max}$

$$ P[X \leq P_{max} ] = 0.75 $$
$$ P_{max} \approx 196.9 $$


\end{enumerate}
\item La cantidad de l\'iquido que una m\'aquina de llenado autom\'atico deposita en latas de una bebida gaseosa tienen una distribuci\'on normal con media 3.67 dec\'ilitros y desviaci\'on est\'andar de 0.03 dec\'ilitros.


\begin{enumerate}[a)]
\item ?`Cu\'al es la probabilidad que el volumen depositado sea menor que 3.55 dec\'ilitros?

$$ P[X \leq 3.55] = 0.00003 $$

\item Si se desechan todas las latas que tienen menos de 3.58 dec\'ilitros o m\'as de 3.73 dec\'ilitros, ?`Cu\'al es el porcentaje de latas desechadas?

$$ P[3.58 \leq X \leq 3.73] \approx 2.4\% $$

\item Calcule las especificaciones de llenado (sim\'etricas alrededor de la media), de modo que se incluya el 99\% de todas las latas.

$$ P[Linf \leq X \leq Lsup] = 1\% $$
$$Lsup = 3.75$$
$$Linf = 3.59$$



\end{enumerate}


\end{enumerate}

\end{document}