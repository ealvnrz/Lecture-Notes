\documentclass{article}
\usepackage[margin=2cm]{geometry}
\usepackage{graphicx}
\usepackage{fancyhdr}
\usepackage{enumerate}
\usepackage{epsfig}
\usepackage[english]{babel}
\usepackage[utf8x]{inputenc}
\usepackage{amsmath,amssymb}
\usepackage{graphics,color, epstopdf}
\setlength{\headheight}{70pt} 
\setlength{\headsep}{20pt} 
\pagestyle{fancy}
\lhead{}
\chead{Universidad de Valparaíso \\ Facultad de Ciencias \\ Instituto de Estadística }
\rhead{Profesor: Eloy Alvarado}

\begin{document}
\begin{center}
\textbf{Ejercicios Test de hipótesis - Proporción y Varianza}
\end{center}
\begin{enumerate}
\item En el año $2016$, el $16\%$ de las embarazadas fueron adolecentes menores de $20$ años. El Ministerio de Salud desea saber si esa cifra ha aumentado, para ello se considera una muestra de $704$ partos ocurridos en el Hospital Metropolitano, donde $132$  de ellos corresponden a embarazadas que son adolecentes menores de $20$ años. Con la información anterior. ¿Es posible aseverar que el porcentaje ha aumentado?
\item Una empresa está interesada en lanzar un nuevo producto al mercado. Tras realizar una campaña publicitaria, se toma una muestra de $1000$ habitantes, de los cuales, $25$ no conocían el producto. A un nivel de significancia del $1\%$, ¿El estudio apoya las siguientes hipótesis?
\begin{enumerate}[a)]
\item Más del $3\%$ de la población no conoce el nuevo producto.
\item Menos del $2\%$ de la población no conoce el nuevo producto.
\end{enumerate}
\item Se supone que los diámetros de cierta marca de válvulas están distribuidos normalmente con una varianza poblacional de $0.2$ pulgadas$^2$, pero se cree que últimamente han aumentado. Se toma una muestra aleatoria de válvulas a las que se les mide su diámetro, obteniéndose los siguientes resultados (en pulgadas):
\begin{center}
$5.5\hspace{10pt}5.4\hspace{10pt}5.4\hspace{10pt}5.6\hspace{10pt}5.8\hspace{10pt}5.4\hspace{10pt}5.5\hspace{10pt}5.4\hspace{10pt}5.6\hspace{10pt}5.7$
\end{center}
Con la información disponible, ¿Es posible aseverar que la varianza ha aumentado?
\item Una empresa del giro alimenticio desea determinar si los lotes de una materia prima tienen o no una varianza poblacional mayor a $15$ en su grado de endulzamiento, para ello se realiza un muestreo de 20 lotes y se obtiene una varianza muestral de $20.98$. ¿Es posible aseverar que la varianza es mayor a $15$ en su grado de endulzamiento?
\end{enumerate}

\end{document}