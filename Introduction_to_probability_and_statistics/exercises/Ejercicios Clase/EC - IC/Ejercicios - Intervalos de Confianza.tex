\documentclass{article}
\usepackage[margin=2cm]{geometry}
\usepackage{graphicx}
\usepackage{fancyhdr}
\usepackage{enumerate}
\usepackage{epsfig}
\usepackage[english]{babel}
\usepackage[utf8x]{inputenc}
\usepackage{amsmath}
\usepackage{graphics,color, epstopdf}
\setlength{\headheight}{70pt} 
\setlength{\headsep}{20pt} 
\pagestyle{fancy}
%\lhead{}
\chead{Universidad de Valparaíso \\ Facultad de Ciencias \\ Instituto de Estadística }
\rhead{Profesor: Eloy Alvarado}

\begin{document}
\begin{center}
\textbf{Ejercicios Estimación por Intervalo}
\end{center}
\begin{enumerate}
\item El índice de resistencia a la rotura, expresado en kg, de un determinado tipo de cuerda sigue una distribución Normal con
desviación típica 15.6 kg. Con una muestra de 5 de estas cuerdas, seleccionadas al azar, se obtuvieron los siguientes índices:
\begin{center}
280, 240, 270, 285, 270.
\end{center}
\begin{enumerate}[a)]
\item Obtenga un intervalo de confianza para la media del índice de resistencia a la rotura de este tipo de cuerdas, utilizando un nivel de
confianza del 95\%.
\item Si, con el mismo nivel de confianza, se desea obtener un error máximo en la estimación de la media de 5 kg, ¿será suficiente con elegir
una muestra de 30 cuerdas?
\end{enumerate}
\item En un hospital se ha tomado la temperatura a una muestra de 64 pacientes para estimar la temperatura media de sus
enfermos. La media de la muestra ha sido 37,1 ºC y se sabe que la desviación típica de toda la población es 1,04 ºC.
\begin{enumerate}[a)]
\item Obtenga un intervalo de confianza, al 90\%, para la media poblacional.
\item ¿Con qué nivel de confianza podemos afirmar que la media de la población está comprendida entre 36,8ºC y 37,4 ºC?
\end{enumerate}
\item Un sondeo de 100 votantes elegidos al azar en un distrito indica que el 55\% de ellos estaban a favor de un cierto candidato. 
\begin{enumerate}[a)]
\item Hallar los límites de confianza (a) 95\% (b) 99\% (c) 99.73\% para la proporción de todos los votantes favorables a ese candidato.
\item ¿De qué tamaño hay tomar el sondeo para tener al 95\% de confianza que el candidato saldrá elegido?
\end{enumerate}
\item Una muestra de 150 lámparas del tipo A ha dado una vida media de 1400 horas y una desviación típica de 120 horas. Una muestra de 200 lámparas del tipo B dan vida media de 1200 horas y desviación típica de 80 horas. Hallar  los límites de confianza (a) 95\% (b) 99\% para la diferencia  de las vidas medias de las poblaciones de ambos tipos.
\end{enumerate}

\end{document}