\documentclass{article}
\usepackage[margin=2cm]{geometry}
\usepackage{graphicx}
\usepackage{amsmath,mathtools}
\usepackage{fancyhdr}
\usepackage{enumerate}
\usepackage{epsfig}
\usepackage{graphics,color, epstopdf}
\setlength{\headheight}{70pt} 
\setlength{\headsep}{20pt} 
\pagestyle{fancy}
\lhead{}
\chead{Universidad de Valpara\'iso \\ Facultad de Ciencias \\ Instituto de Estad\'istica }
\rhead{Profesor: Eloy Alvarado}

\begin{document}
\begin{center}
\textbf{Ejercicios Clase, Variables Aleatorias Continuas}
\end{center}
\begin{enumerate}
\item Una variable aleatoria $X$ tiene funci\'on de densidad:
\[ 
f(x)= \begin{dcases}
\dfrac{c}{x^2+1} , & -\infty < x < \infty \\
0 ,& e.o.c
\end{dcases}
\]
\begin{enumerate}[(a)]
\item Hallar el valor de la constante c.
\item Hallar la probabilidad de que $X^2$ est\'e entre $\dfrac{1}{3}$ y $1$.

Revisar pdf p.96

\end{enumerate}
\item La funci\'on de distribuci\'on para una variable aleatoria $X$ es:
\[
F(x)= \begin{dcases}
1-e^{-2x}, & x\geq 0 \\
0, & x < 0
\end{dcases}
\]

\begin{enumerate}[(a)]
\item Hallar la funci\'on de densidad.

$$f(x) = F'(x) = e^{-2 xº}$$

\item Hallar la probabilidad de que $X>2$.

$$ P(X \geq 2) = 1 - P(X \leq 2) \approx 0.982 $$

\item Hallar la probabilidad de que $-3<X\leq4$.

$$ P(-3 \leq X \geq 4) =P(0 \leq X \leq 4) = P( X \leq 4) = 0.999$$


\end{enumerate}
\item Una persona jugando a los dardos encuentra que la probabilidad de que el dardo caiga entre $r$ y $r+dr$ es:\\

{\centering $\mathrm{P}(r \leq R \leq r+dr)=c[1-(\dfrac{r}{a})^2]dr$ \\}
\hspace{2px}\\
Aqu\'i, R es la distancia del impacto desde el centro del objetivo, $c$ es una constante y $a$  es el radio del objetivo. Hallar la probabilidad de pegar en el blanco, que se supone tiene radio $b$. Suponer que siempre se hace impacto en el objetivo.

$$P[X \leq b] = \int_0^b c [1-r^2/a^2] dr = c \left[  b- \dfrac{b^3}{3 a^2}  \right]  $$


\item Una variable aleatoria $X$ tiene funci\'on de densidad:\\

$$f(x)= \begin{dcases}
cx^2, & 1\leq x\leq2 \\
cx, & 2<x<3 \\
0, & e.o.c
\end{dcases}$$
\begin{enumerate}[(a)]
\item Hallar la constante $c$.
\item Hallar $P(X>2)$
\item Hallar $P(\frac{1}{2} < X < \frac{3}{2})$
\item Hallar $E(X)$, $V(X)$
\item Se define una variable aleatoria $Y=3X+2$. Encontrar $E(Y)$, $V(Y)$
\end{enumerate}

\item Sup\'ongase que cierta pieza met\'alica se romper\'a despu\'es de sufrir dos ciclos de esfuerzo. Si estos ciclos ocurren de manera independiente a una frecuencia promedio de dos por cada 100 horas, obtener la probabilidad de que el intervalo de tiempo se encuentre hasta que ocurre el segundo ciclo:
\begin{enumerate}[a)]
\item dentro de una desviaci\'on est\'andar del tiempo promedio.

El evento de la ocurrencia de los 2 ciclos de esfuerzo lo definimos por la variable aleatoria $X$, por lo tanto
$$P[X\leq 50]$$

Con una media y desviación estándar de

$$E[X]= 50$$
$$(Var[X])^{0.5}=50$$

Esta información la utilizamos para definir la variable aleatoria de conteo (Poisson) que está relacionada.
$$N_{50} = Y = Poisson(\dfrac{1}{50}*50)$$
$$N_{50} = Y = Poisson(1)$$

Ahora
$$1 - P(Y=1) - P(Y=0)=0.2642$$

\item a m\'as de dos desviaciones est\'andar por encima de la media.

$$P[X \geq 150] = 1-P[X \leq 150]$$

De manera similar al punto anterior definmos 

$$N_{150} = Y = Poisson(\dfrac{1}{50}*150)$$
$$N_{150} = Y = Poisson(3)$$

Ahora
$$1 - P(Y=0)- P(Y=1) -P(Y=2)=0.577$$

\end{enumerate}
\newpage
\item En cierta ciudad el consumo diario de energ\'ia el\'ectrica, en millones de kilovatios por hora, puede considerarse como una variable aleatoria con distribuci\'on Gamma de par\'ametros: $\alpha =3$ y $\lambda = 0.5$. La planta de energ\'ia de esta ciudad tiene una capacidad diaria de 10 millones de KW / hora. ?` Cu\'al es la probabilidad de que este abastecimiento sea:
\begin{enumerate}[a)]
\item Insuficiente en un d\'ia cualquiera.

$$P(X > 10)= 1 -P(X \leq 10)$$

\item Se consuman entre 3 y 8 millones de KW/hora.
\item Encuentre el consumo esperado en un d\'ia cualquiera.
\end{enumerate}
\item \textbf{(Propuesto)} La edad a la que un hombre contrae matrimonio por primera vez es una variable aleatoria con distribuci\'on Gamma. Si la edad promedio es de 30 a\~nos y lo m\'as com\'un es que el hombre se case a los 23 a\~nos, encontrar los valores de los par\'ametros $\alpha$ y $\lambda$, para esta distribuci\'on.
\item Supongo que un sistema contiene cierto tipo de componente cuyo tiempo de falla en a\~nos est\'a dado por  la variable aleatoria T, distribuida exponencialmente con tiempo promedio de falla $\beta=5$ $( \beta= \dfrac{1}{\lambda})$. Si 5 de estos componentes se instalan en diferentes sistemas. ?` Cu\'al es la probabilidad de que al menos 2 contin\'uen funcionando despu\'es de 8 a\~nos?.

\item  Se ha comprobado que el tiempo de vida de cierto tipo de marcapasos sigue una distribuci\'on  exponencial con media de 16 a\~nos. ?`Cu\'al es la probabilidad de que a una persona a la que se le ha implantado este marcapasos se le deba reimplantar otro antes de 20 a\~nos? Si el marcapasos lleva funcionando correctamente 5 a\~nos en un paciente, ?`Cu\'al es la probabilidad de que haya que cambiarlo antes de 25\% a\~nos?
\end{enumerate}
\end{document}