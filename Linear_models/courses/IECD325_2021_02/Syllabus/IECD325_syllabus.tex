\documentclass[11pt,]{article}
\usepackage[margin=1in]{geometry}
\newcommand*{\authorfont}{\fontfamily{phv}\selectfont}
\usepackage[]{mathpazo}
\usepackage{abstract}
\renewcommand{\abstractname}{}    % clear the title
\renewcommand{\absnamepos}{empty} % originally center
\newcommand{\blankline}{\quad\pagebreak[2]}

\providecommand{\tightlist}{%
  \setlength{\itemsep}{0pt}\setlength{\parskip}{0pt}} 
\usepackage{longtable,booktabs}

\usepackage{parskip}
\usepackage{titlesec}
\titlespacing\section{0pt}{12pt plus 4pt minus 2pt}{6pt plus 2pt minus 2pt}
\titlespacing\subsection{0pt}{12pt plus 4pt minus 2pt}{6pt plus 2pt minus 2pt}

\titleformat*{\subsubsection}{\normalsize\itshape}

\usepackage{titling}
\setlength{\droptitle}{-.25cm}

%\setlength{\parindent}{0pt}
%\setlength{\parskip}{6pt plus 2pt minus 1pt}
%\setlength{\emergencystretch}{3em}  % prevent overfull lines 

\usepackage[T1]{fontenc}
\usepackage[utf8]{inputenc}
\usepackage[spanish]{babel}
\usepackage{fancyhdr}
\pagestyle{fancy}
\usepackage{lastpage}
\renewcommand{\headrulewidth}{0.3pt}
\renewcommand{\footrulewidth}{0.0pt} 
\lhead{}
\chead{}
\rhead{\footnotesize IECD 325: Modelos Lineales y Diseño de Experimentos -- 2021/02}
\lfoot{}
\cfoot{\small \thepage/\pageref*{LastPage}}
\rfoot{}

\fancypagestyle{firststyle}
{
\renewcommand{\headrulewidth}{0pt}%
   \fancyhf{}
   \fancyfoot[C]{\small \thepage/\pageref*{LastPage}}
}

%\def\labelitemi{--}
%\usepackage{enumitem}
%\setitemize[0]{leftmargin=25pt}
%\setenumerate[0]{leftmargin=25pt}




\makeatletter
\@ifpackageloaded{hyperref}{}{%
\ifxetex
  \usepackage[setpagesize=false, % page size defined by xetex
              unicode=false, % unicode breaks when used with xetex
              xetex]{hyperref}
\else
  \usepackage[unicode=true]{hyperref}
\fi
}
\@ifpackageloaded{color}{
    \PassOptionsToPackage{usenames,dvipsnames}{color}
}{%
    \usepackage[usenames,dvipsnames]{color}
}
\makeatother
\hypersetup{breaklinks=true,
            bookmarks=true,
            pdfauthor={ ()},
             pdfkeywords = {},  
            pdftitle={IECD 325: Modelos Lineales y Diseño de Experimentos},
            colorlinks=true,
            citecolor=blue,
            urlcolor=blue,
            linkcolor=magenta,
            pdfborder={0 0 0}}
\urlstyle{same}  % don't use monospace font for urls


\setcounter{secnumdepth}{0}





\usepackage{setspace}

\title{IECD 325: Modelos Lineales y Diseño de Experimentos}
\author{Eloy Alvarado Narváez}
\date{2021/02}


\begin{document}  

		\maketitle
		
	
		\thispagestyle{firststyle}

%	\thispagestyle{empty}


	\noindent \begin{tabular*}{\textwidth}{ @{\extracolsep{\fill}} lr @{\extracolsep{\fill}}}


E-mail: \texttt{\href{mailto:eloy.alvarado@uv.cl}{\nolinkurl{eloy.alvarado@uv.cl}}} & Web: \href{http://classroom.google.com/c/MzgyMjQ3MTkyMzQ4?cjc=3kkm3hg}{\tt Google Classroom }\\
Horario de atención: online  &  Horario de Clase: MA 6 - 7.30 PM; MI 8.30 - 11.45 AM; JU 8.30 - 10.00\\
Oficina: Oficina  & Sala de Clases: \emph{online}\\
	&  \\
	\hline
	\end{tabular*}
	
\vspace{2mm}
	


\hypertarget{descripciuxf3n-de-la-asignatura}{%
\section{Descripción de la
asignatura}\label{descripciuxf3n-de-la-asignatura}}

Proporcionar los elementos básicos de regresión lineal y diseño de
experimentos. Al término del curso el alumno debe ser capaz de:

\begin{enumerate}
\def\labelenumi{\arabic{enumi}.}
\tightlist
\item
  Reconocer y plantear modelos lineales.
\item
  Estimar parámetros y realizar inferencia.
\item
  Analizar y verificar los supuestos de un modelo.
\item
  Seleccionar el mejor modelo.
\item
  Diseñar el experimento estadístico más apropiado.
\item
  Determinar tamaños muestrales para los diseños que se postulan.
\item
  Elaborar informes que concluyan metodología y resultados
  interactuando, donde sea posible, con software que se disponga.
\end{enumerate}

\hypertarget{aporte-al-perfil-de-egreso}{%
\section{Aporte al Perfil de Egreso}\label{aporte-al-perfil-de-egreso}}

Indicar Competencia (s) del perfil de egreso de Licenciatura y/o Título
Profesional al que apunta la asignatura con su respectivo nivel de
dominio:

\begin{itemize}
\tightlist
\item
  CE1: Resuelve problemas en diferentes áreas del conocimiento,
  utilizando teoría, métodos y técnicas de análisis estadísticos
  pertinentes, en especial, en su área minor para asesorar.
\item
  ND2: Analiza problemas de alta y baja complejidad que involucran datos
  con insipiente nivel de independencia.
\item
  CE2: Integra tecnología computacional en distintos escenarios de
  complejidad para realizar análisis estadísticos.
\item
  ND2: Opera con propiedad programas computacionales estadísticos y
  herramientas para el análisis de datos complejos.
\item
  CE3: Fundamenta teóricamente el uso apropiado de metodologías
  estadísticas y de ciencia de datos para sustentar los resultados
  obtenidos.
\item
  ND2: Explica y aplica diferentes metodologías de la estadística
  teórica y avanzada. Indicar Competencia Genérica del perfil de egreso
  al que apunta la asignatura indicando nivel de desempeño:
\item
  CG2: Utiliza en forma responsable sus conocimientos, considerando las
  implicancias éticas de su accionar en las personas, la sociedad y el
  medio.
\item
  ND2: Demuestra comportamientos éticos asociados a la responsabilidad
  ciudadana en contextos académicos y de vinculación con el medio.
\item
  CG3: Emplea de manera correcta y pertinente el idioma español, de
  forma oral y escrita, para un adecuado desenvolvimiento profesional.
  Además, comprende, utiliza y analiza textos escritos en inglés con
  propósitos académicos que contribuyan al desarrollo profesional.
\item
  ND2: Desarrolla habilidades de comunicación interpersonal, para el
  desempeño satisfactorio en el trabajo académico y en distintos
  contextos socioculturales.
\end{itemize}

\hypertarget{resultados-de-aprendizaje-y-desempeuxf1os}{%
\section{Resultados de aprendizaje y
desempeños:}\label{resultados-de-aprendizaje-y-desempeuxf1os}}

Al final de la asignatura los estudiantes serán capaces de demostrar los
siguientes resultados de aprendizaje del segundo nivel de dominio de las
Competencias Específicas del perfil de egreso a las que apunta la
asignatura tanto en conocimientos, habilidades y/o actitudes:

\begin{itemize}
\tightlist
\item
  CE1-ND2-RA1: Diferencia diversos modelos estadísticos para su adecuada
  aplicación.
\item
  CE1-ND2-RA2: Diseña procedimientos de captura de datos de acuerdo a
  las características del problema a resolver.
\item
  CE2-ND2-RA2: Utiliza eficientemente herramientas de programación y
  software estadísticos para análisis de complejidad intermedia de
  datos.
\item
  CE3-ND2-RA1: Distingue diversas metodologías o herramientas
  estadísticas que se pueden aplicar para el análisis de complejidad
  intermedia de datos. Al final de la asignatura los estudiantes serán
  capaces de demostrar los siguientes desempeños tanto en conocimientos,
  habilidades y/o actitudes de las Competencias Genéricas del Perfil de
  Egreso a las que apunta la asignatura:
\item
  CG2-ND2-DC1: Implementa actividades en vinculación con el medio,
  demostrando comportamientos éticos, contemplando las necesidades e
  intereses de la comunidad local y/o regional.
\item
  CG3-ND2-DC1: Expresa con claridad las propias necesidades y
  requerimientos en el trabajo colaborativo en contextos académicos y
  socioculturales.
\end{itemize}

\hypertarget{bibliografuxeda-principal}{%
\section{Bibliografía principal}\label{bibliografuxeda-principal}}

Belsley, David A, Edwin Kuh, and Roy E Welsch (2005).
\emph{Regression diagnostics: Identifying influential data and sources of collinearity}.
Vol. 571. John Wiley \& Sons.

Box, George EP, J Stuart Hunter, William Gordon Hunter, and others
(2005).
\emph{Statistics for experimenters: design, innovation, and discovery}.
Vol. 2. Wiley-Interscience New York.

Dean, Angela, Daniel Voss, Danel Draguljic, and others (1999).
\emph{Design and analysis of experiments}. Vol. 1. Springer.

Faraway, Julian J (2004). \emph{Linear models with R}. Chapman and
Hall/CRC.

Lawson, John (2014). \emph{Design and Analysis of Experiments with R}.
Vol. 115. CRC press.

Madsen, Henrik and Poul Thyregod (2010).
\emph{Introduction to general and generalized linear models}. CRC Press.

Montgomery, Douglas C (2005). \emph{Diseño y análisis de experimentos.}
Limusa Wiley.

Rencher, Alvin C and G Bruce Schaalje (2008).
\emph{Linear models in statistics}. John Wiley \& Sons.

Seber, George AF and Alan J Lee (2012).
\emph{Linear regression analysis}. Vol. 329. John Wiley \& Sons.

Weisberg, Sanford (2005). \emph{Applied linear regression}. Vol. 528.
John Wiley \& Sons.

\hypertarget{evaluaciones} Tarea 1.
\item
  \textbf{30\%} Prueba 1.
\item
  \textbf{10\%} Tarea 2.
\item
  \textbf{30\%} Prueba 2.
\item
  \textbf{20\%} Trabajo Final
\end{itemize}

La nota mínina de aprobación es un 4.0. Si la nota de presentación está
entre 3.5 y 4.5 se deberá rendir un examen, con ponderación 70\% Nota de
presentación, 30\% examen.

Las evaluaciones deberán ser subidas al módulo de la clase entre los
plazos determinados. Si existiera un retraso en la entrega, se
descontará puntaje progresivamente.

\newpage

\hypertarget{calendarizaciuxf3n}{%
\section{Calendarización}\label{calendarizaciuxf3n}}

\hypertarget{semana-01-0609---1009-presentaciuxf3n-de-la-asignatura}{%
\subsection{Semana 01, 06/09 - 10/09: Presentación de la
asignatura}\label{semana-01-0609---1009-presentaciuxf3n-de-la-asignatura}}

\begin{itemize}
\tightlist
\item
  Relaciones entre variables
\item
  Modelo lineal simple
\item
  Estimación de parámetros
\item
  Método de los mínimos cuadrados
\item
  Teorema de Gauss-Markov.
\end{itemize}

\hypertarget{semana-02-1309---1709-receso-fiestas-patrias}{%
\subsection{Semana 02, 13/09 - 17/09: Receso fiestas
patrias}\label{semana-02-1309---1709-receso-fiestas-patrias}}

\hypertarget{semana-03-2009---2409-regresiuxf3n-lineal-simple}{%
\subsection{Semana 03, 20/09 - 24/09: Regresión Lineal
Simple}\label{semana-03-2009---2409-regresiuxf3n-lineal-simple}}

\begin{itemize}
\tightlist
\item
  Inferencia sobre los parámetros: Estimación por intervalos y prueba de
  hipótesis
\item
  Estimación de la respuesta media y predicción
\item
  Análisis de varianza: Tabla ANDEVA y coeficiente de determinación.
\end{itemize}

\hypertarget{semana-04-2709---0110-regresiuxf3n-lineal-muxfaltiple}{%
\subsection{Semana 04, 27/09 - 01/10: Regresión Lineal
Múltiple}\label{semana-04-2709---0110-regresiuxf3n-lineal-muxfaltiple}}

\begin{itemize}
\tightlist
\item
  Modelo de regresión múltiple

  \begin{itemize}
  \tightlist
  \item
    Estimación mínima cuadrática. Propiedades. Teorema de Gauss-Markov
  \item
    Distribución del EMC y otros temas relacionados.
  \end{itemize}
\item
  Mínimos cuadrados ponderados
\item
  Estimación con restricciones lineales
\end{itemize}

\hypertarget{semana-05-0410---0810-regresiuxf3n-lineal-muxfaltiple}{%
\subsection{Semana 05, 04/10 - 08/10: Regresión Lineal
Múltiple}\label{semana-05-0410---0810-regresiuxf3n-lineal-muxfaltiple}}

\begin{itemize}
\tightlist
\item
  Estimación por Intervalos

  \begin{itemize}
  \tightlist
  \item
    Intervalo de confianza para \(\beta\) y una función lineal de los
    \(\beta\)
  \end{itemize}
\item
  Predicción
\item
  Dócimas de hipótesis: test t, ANOVA, test F
\item
  Test de bondad de ajuste. Replicaciones. Falta de ajuste
\end{itemize}

\hypertarget{semana-06-1110---1510-regresiuxf3n-lineal-muxfaltiple-y-anuxe1lisis-de-supuestos}{%
\subsection{Semana 06, 11/10 - 15/10: Regresión Lineal Múltiple y
Análisis de
supuestos}\label{semana-06-1110---1510-regresiuxf3n-lineal-muxfaltiple-y-anuxe1lisis-de-supuestos}}

\begin{itemize}
\tightlist
\item
  Variables indicadoras
\item
  Regresión polinomial: polinomios ortogonales
\item
  Selección del modelo. Eliminación hacia atrás, selección hacia
  adelante y regresión paso a paso
\item
  \textbf{Publicación Tarea \#1}.
\item
  Analisis de residuos. Gráficos residuales
\item
  Análisis de supuestos de normalidad
\end{itemize}

\hypertarget{semana-07-1810---2210-anuxe1lisis-de-supuestos}{%
\subsection{Semana 07, 18/10 - 22/10: Análisis de
supuestos}\label{semana-07-1810---2210-anuxe1lisis-de-supuestos}}

\begin{itemize}
\tightlist
\item
  Test de Durbin-Watson
\item
  Transformaciones
\item
  Detección de casos influyentes
\item
  Multicolinealidad. Consecuencias, detección y medidas remediales.
\item
  \textbf{Entrega Tarea \#1}.
\end{itemize}

\hypertarget{semana-08-2510---2910-ejercitaciuxf3n-y-prueba-1}{%
\subsection{Semana 08, 25/10 - 29/10: Ejercitación y Prueba
\#1}\label{semana-08-2510---2910-ejercitaciuxf3n-y-prueba-1}}

\begin{itemize}
\tightlist
\item
  Ejercicios Prueba \#1
\item
  Consultas
\item
  \textbf{Prueba \#1}.
\end{itemize}

\hypertarget{semana-09-0111---0511-pausa-activa}{%
\subsection{Semana 09, 01/11 - 05/11: Pausa
Activa}\label{semana-09-0111---0511-pausa-activa}}

\hypertarget{semana-10-0811---1211-modelos-buxe1sicos-de-diseuxf1o-de-experimentos}{%
\subsection{Semana 10, 08/11 - 12/11: Modelos básicos de diseño de
experimentos}\label{semana-10-0811---1211-modelos-buxe1sicos-de-diseuxf1o-de-experimentos}}

\begin{itemize}
\tightlist
\item
  Revisión Prueba \#1
\item
  Diseños completamente aleatorizados
\item
  Experimentos a un factor con efectos fijo (one-way ANOVA)
\item
  Experimentos a un factor con efecto aleatorio
\item
  Comparaciones múltiples
\item
  Experimentos a más de un factor con efectos fijos (two-way ANOVA).
\end{itemize}

\hypertarget{semana-11-1511---1911-diseuxf1o-en-bloques.}{%
\subsection{Semana 11, 15/11 - 19/11: Diseño en
bloques.}\label{semana-11-1511---1911-diseuxf1o-en-bloques.}}

\begin{itemize}
\tightlist
\item
  Diseño en bloques completos al azar
\item
  Diseño cuadrado latino
\item
  Diseño cuadrado greco-latino
\item
  Diseño por bloques incompletos
\end{itemize}

\hypertarget{semana-12-2211---2611-experimentos-factoriales}{%
\subsection{Semana 12, 22/11 - 26/11: Experimentos
factoriales}\label{semana-12-2211---2611-experimentos-factoriales}}

\begin{itemize}
\tightlist
\item
  Introducción
\item
  Factorial \(2^k\)
\item
  Técnica de confusión en el diseño factorial \(2^k\)
\item
  \textbf{Publicación Tarea \#2}.
\end{itemize}

\hypertarget{semana-13-2911---0312-otros-enfoques-de-diseuxf1os-experimentales}{%
\subsection{Semana 13, 29/11 - 03/12: Otros enfoques de diseños
experimentales}\label{semana-13-2911---0312-otros-enfoques-de-diseuxf1os-experimentales}}

\begin{itemize}
\tightlist
\item
  Diseños jerárquicos o anidados
\item
  Modelos mixtos
\item
  Análisis de Covarianza.
\item
  \textbf{Entrega Tarea \#2}.
\end{itemize}

\hypertarget{semana-14-0612---1012-ejercitaciuxf3n-y-prueba-2}{%
\subsection{Semana 14, 06/12 - 10/12: Ejercitación y Prueba
\#2}\label{semana-14-0612---1012-ejercitaciuxf3n-y-prueba-2}}

\begin{itemize}
\tightlist
\item
  Ejercicios Prueba \#2
\item
  Consultas
\item
  \textbf{Prueba \#2}.
\item
  \textbf{Publicación trabajo final}
\end{itemize}

\hypertarget{semana-15-1312---1712-trabajo-final}{%
\subsection{Semana 15, 13/12 - 17/12: Trabajo
final}\label{semana-15-1312---1712-trabajo-final}}

\begin{itemize}
\item
  Revisión Prueba \#2
\item
  Presentaciones trabajo final
\end{itemize}

\hypertarget{semana-16-2012---2412-prueba-recuperativa-y-examen}{%
\subsection{Semana 16, 20/12 - 24/12: Prueba Recuperativa y
examen}\label{semana-16-2012---2412-prueba-recuperativa-y-examen}}

\begin{itemize}
\tightlist
\item
  Cierre de semestre
\end{itemize}




\end{document}

\makeatletter
\def\@maketitle{%
  \newpage
%  \null
%  \vskip 2em%
%  \begin{center}%
  \let \footnote \thanks
    {\fontsize{18}{20}\selectfont\raggedright  \setlength{\parindent}{0pt} \@title \par}%
}
%\fi
\makeatother
